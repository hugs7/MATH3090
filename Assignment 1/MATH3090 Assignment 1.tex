\documentclass[11pt]{article}

    \usepackage[breakable]{tcolorbox}
    \usepackage{parskip} % Stop auto-indenting (to mimic markdown behaviour)
    

    % Basic figure setup, for now with no caption control since it's done
    % automatically by Pandoc (which extracts ![](path) syntax from Markdown).
    \usepackage{graphicx}
    % Keep aspect ratio if custom image width or height is specified
    \setkeys{Gin}{keepaspectratio}
    % Maintain compatibility with old templates. Remove in nbconvert 6.0
    \let\Oldincludegraphics\includegraphics
    % Ensure that by default, figures have no caption (until we provide a
    % proper Figure object with a Caption API and a way to capture that
    % in the conversion process - todo).
    \usepackage{caption}
    \DeclareCaptionFormat{nocaption}{}
    \captionsetup{format=nocaption,aboveskip=0pt,belowskip=0pt}

    \usepackage{float}
    \floatplacement{figure}{H} % forces figures to be placed at the correct location
    \usepackage{xcolor} % Allow colors to be defined
    \usepackage{enumerate} % Needed for markdown enumerations to work
    \usepackage{geometry} % Used to adjust the document margins
    \usepackage{amsmath} % Equations
    \usepackage{amssymb} % Equations
    \usepackage{textcomp} % defines textquotesingle
    % Hack from http://tex.stackexchange.com/a/47451/13684:
    \AtBeginDocument{%
        \def\PYZsq{\textquotesingle}% Upright quotes in Pygmentized code
    }
    \usepackage{upquote} % Upright quotes for verbatim code
    \usepackage{eurosym} % defines \euro

    \usepackage{iftex}
    \ifPDFTeX
        \usepackage[T1]{fontenc}
        \IfFileExists{alphabeta.sty}{
              \usepackage{alphabeta}
          }{
              \usepackage[mathletters]{ucs}
              \usepackage[utf8x]{inputenc}
          }
    \else
        \usepackage{fontspec}
        \usepackage{unicode-math}
    \fi

    \usepackage{fancyvrb} % verbatim replacement that allows latex
    \usepackage{grffile} % extends the file name processing of package graphics
                         % to support a larger range
    \makeatletter % fix for old versions of grffile with XeLaTeX
    \@ifpackagelater{grffile}{2019/11/01}
    {
      % Do nothing on new versions
    }
    {
      \def\Gread@@xetex#1{%
        \IfFileExists{"\Gin@base".bb}%
        {\Gread@eps{\Gin@base.bb}}%
        {\Gread@@xetex@aux#1}%
      }
    }
    \makeatother
    \usepackage[Export]{adjustbox} % Used to constrain images to a maximum size
    \adjustboxset{max size={0.9\linewidth}{0.9\paperheight}}

    % The hyperref package gives us a pdf with properly built
    % internal navigation ('pdf bookmarks' for the table of contents,
    % internal cross-reference links, web links for URLs, etc.)
    \usepackage{hyperref}
    % The default LaTeX title has an obnoxious amount of whitespace. By default,
    % titling removes some of it. It also provides customization options.
    \usepackage{titling}
    \usepackage{longtable} % longtable support required by pandoc >1.10
    \usepackage{booktabs}  % table support for pandoc > 1.12.2
    \usepackage{array}     % table support for pandoc >= 2.11.3
    \usepackage{calc}      % table minipage width calculation for pandoc >= 2.11.1
    \usepackage[inline]{enumitem} % IRkernel/repr support (it uses the enumerate* environment)
    \usepackage[normalem]{ulem} % ulem is needed to support strikethroughs (\sout)
                                % normalem makes italics be italics, not underlines
    \usepackage{soul}      % strikethrough (\st) support for pandoc >= 3.0.0
    \usepackage{mathrsfs}
    

    
    % Colors for the hyperref package
    \definecolor{urlcolor}{rgb}{0,.145,.698}
    \definecolor{linkcolor}{rgb}{.71,0.21,0.01}
    \definecolor{citecolor}{rgb}{.12,.54,.11}

    % ANSI colors
    \definecolor{ansi-black}{HTML}{3E424D}
    \definecolor{ansi-black-intense}{HTML}{282C36}
    \definecolor{ansi-red}{HTML}{E75C58}
    \definecolor{ansi-red-intense}{HTML}{B22B31}
    \definecolor{ansi-green}{HTML}{00A250}
    \definecolor{ansi-green-intense}{HTML}{007427}
    \definecolor{ansi-yellow}{HTML}{DDB62B}
    \definecolor{ansi-yellow-intense}{HTML}{B27D12}
    \definecolor{ansi-blue}{HTML}{208FFB}
    \definecolor{ansi-blue-intense}{HTML}{0065CA}
    \definecolor{ansi-magenta}{HTML}{D160C4}
    \definecolor{ansi-magenta-intense}{HTML}{A03196}
    \definecolor{ansi-cyan}{HTML}{60C6C8}
    \definecolor{ansi-cyan-intense}{HTML}{258F8F}
    \definecolor{ansi-white}{HTML}{C5C1B4}
    \definecolor{ansi-white-intense}{HTML}{A1A6B2}
    \definecolor{ansi-default-inverse-fg}{HTML}{FFFFFF}
    \definecolor{ansi-default-inverse-bg}{HTML}{000000}

    % common color for the border for error outputs.
    \definecolor{outerrorbackground}{HTML}{FFDFDF}

    % commands and environments needed by pandoc snippets
    % extracted from the output of `pandoc -s`
    \providecommand{\tightlist}{%
      \setlength{\itemsep}{0pt}\setlength{\parskip}{0pt}}
    \DefineVerbatimEnvironment{Highlighting}{Verbatim}{commandchars=\\\{\}}
    % Add ',fontsize=\small' for more characters per line
    \newenvironment{Shaded}{}{}
    \newcommand{\KeywordTok}[1]{\textcolor[rgb]{0.00,0.44,0.13}{\textbf{{#1}}}}
    \newcommand{\DataTypeTok}[1]{\textcolor[rgb]{0.56,0.13,0.00}{{#1}}}
    \newcommand{\DecValTok}[1]{\textcolor[rgb]{0.25,0.63,0.44}{{#1}}}
    \newcommand{\BaseNTok}[1]{\textcolor[rgb]{0.25,0.63,0.44}{{#1}}}
    \newcommand{\FloatTok}[1]{\textcolor[rgb]{0.25,0.63,0.44}{{#1}}}
    \newcommand{\CharTok}[1]{\textcolor[rgb]{0.25,0.44,0.63}{{#1}}}
    \newcommand{\StringTok}[1]{\textcolor[rgb]{0.25,0.44,0.63}{{#1}}}
    \newcommand{\CommentTok}[1]{\textcolor[rgb]{0.38,0.63,0.69}{\textit{{#1}}}}
    \newcommand{\OtherTok}[1]{\textcolor[rgb]{0.00,0.44,0.13}{{#1}}}
    \newcommand{\AlertTok}[1]{\textcolor[rgb]{1.00,0.00,0.00}{\textbf{{#1}}}}
    \newcommand{\FunctionTok}[1]{\textcolor[rgb]{0.02,0.16,0.49}{{#1}}}
    \newcommand{\RegionMarkerTok}[1]{{#1}}
    \newcommand{\ErrorTok}[1]{\textcolor[rgb]{1.00,0.00,0.00}{\textbf{{#1}}}}
    \newcommand{\NormalTok}[1]{{#1}}

    % Additional commands for more recent versions of Pandoc
    \newcommand{\ConstantTok}[1]{\textcolor[rgb]{0.53,0.00,0.00}{{#1}}}
    \newcommand{\SpecialCharTok}[1]{\textcolor[rgb]{0.25,0.44,0.63}{{#1}}}
    \newcommand{\VerbatimStringTok}[1]{\textcolor[rgb]{0.25,0.44,0.63}{{#1}}}
    \newcommand{\SpecialStringTok}[1]{\textcolor[rgb]{0.73,0.40,0.53}{{#1}}}
    \newcommand{\ImportTok}[1]{{#1}}
    \newcommand{\DocumentationTok}[1]{\textcolor[rgb]{0.73,0.13,0.13}{\textit{{#1}}}}
    \newcommand{\AnnotationTok}[1]{\textcolor[rgb]{0.38,0.63,0.69}{\textbf{\textit{{#1}}}}}
    \newcommand{\CommentVarTok}[1]{\textcolor[rgb]{0.38,0.63,0.69}{\textbf{\textit{{#1}}}}}
    \newcommand{\VariableTok}[1]{\textcolor[rgb]{0.10,0.09,0.49}{{#1}}}
    \newcommand{\ControlFlowTok}[1]{\textcolor[rgb]{0.00,0.44,0.13}{\textbf{{#1}}}}
    \newcommand{\OperatorTok}[1]{\textcolor[rgb]{0.40,0.40,0.40}{{#1}}}
    \newcommand{\BuiltInTok}[1]{{#1}}
    \newcommand{\ExtensionTok}[1]{{#1}}
    \newcommand{\PreprocessorTok}[1]{\textcolor[rgb]{0.74,0.48,0.00}{{#1}}}
    \newcommand{\AttributeTok}[1]{\textcolor[rgb]{0.49,0.56,0.16}{{#1}}}
    \newcommand{\InformationTok}[1]{\textcolor[rgb]{0.38,0.63,0.69}{\textbf{\textit{{#1}}}}}
    \newcommand{\WarningTok}[1]{\textcolor[rgb]{0.38,0.63,0.69}{\textbf{\textit{{#1}}}}}


    % Define a nice break command that doesn't care if a line doesn't already
    % exist.
    \def\br{\hspace*{\fill} \\* }
    % Math Jax compatibility definitions
    \def\gt{>}
    \def\lt{<}
    \let\Oldtex\TeX
    \let\Oldlatex\LaTeX
    \renewcommand{\TeX}{\textrm{\Oldtex}}
    \renewcommand{\LaTeX}{\textrm{\Oldlatex}}
    % Document parameters
    % Document title
    \title{MATH3090 Assignment 1}
    
    
    
    
    
    
    
% Pygments definitions
\makeatletter
\def\PY@reset{\let\PY@it=\relax \let\PY@bf=\relax%
    \let\PY@ul=\relax \let\PY@tc=\relax%
    \let\PY@bc=\relax \let\PY@ff=\relax}
\def\PY@tok#1{\csname PY@tok@#1\endcsname}
\def\PY@toks#1+{\ifx\relax#1\empty\else%
    \PY@tok{#1}\expandafter\PY@toks\fi}
\def\PY@do#1{\PY@bc{\PY@tc{\PY@ul{%
    \PY@it{\PY@bf{\PY@ff{#1}}}}}}}
\def\PY#1#2{\PY@reset\PY@toks#1+\relax+\PY@do{#2}}

\@namedef{PY@tok@w}{\def\PY@tc##1{\textcolor[rgb]{0.73,0.73,0.73}{##1}}}
\@namedef{PY@tok@c}{\let\PY@it=\textit\def\PY@tc##1{\textcolor[rgb]{0.24,0.48,0.48}{##1}}}
\@namedef{PY@tok@cp}{\def\PY@tc##1{\textcolor[rgb]{0.61,0.40,0.00}{##1}}}
\@namedef{PY@tok@k}{\let\PY@bf=\textbf\def\PY@tc##1{\textcolor[rgb]{0.00,0.50,0.00}{##1}}}
\@namedef{PY@tok@kp}{\def\PY@tc##1{\textcolor[rgb]{0.00,0.50,0.00}{##1}}}
\@namedef{PY@tok@kt}{\def\PY@tc##1{\textcolor[rgb]{0.69,0.00,0.25}{##1}}}
\@namedef{PY@tok@o}{\def\PY@tc##1{\textcolor[rgb]{0.40,0.40,0.40}{##1}}}
\@namedef{PY@tok@ow}{\let\PY@bf=\textbf\def\PY@tc##1{\textcolor[rgb]{0.67,0.13,1.00}{##1}}}
\@namedef{PY@tok@nb}{\def\PY@tc##1{\textcolor[rgb]{0.00,0.50,0.00}{##1}}}
\@namedef{PY@tok@nf}{\def\PY@tc##1{\textcolor[rgb]{0.00,0.00,1.00}{##1}}}
\@namedef{PY@tok@nc}{\let\PY@bf=\textbf\def\PY@tc##1{\textcolor[rgb]{0.00,0.00,1.00}{##1}}}
\@namedef{PY@tok@nn}{\let\PY@bf=\textbf\def\PY@tc##1{\textcolor[rgb]{0.00,0.00,1.00}{##1}}}
\@namedef{PY@tok@ne}{\let\PY@bf=\textbf\def\PY@tc##1{\textcolor[rgb]{0.80,0.25,0.22}{##1}}}
\@namedef{PY@tok@nv}{\def\PY@tc##1{\textcolor[rgb]{0.10,0.09,0.49}{##1}}}
\@namedef{PY@tok@no}{\def\PY@tc##1{\textcolor[rgb]{0.53,0.00,0.00}{##1}}}
\@namedef{PY@tok@nl}{\def\PY@tc##1{\textcolor[rgb]{0.46,0.46,0.00}{##1}}}
\@namedef{PY@tok@ni}{\let\PY@bf=\textbf\def\PY@tc##1{\textcolor[rgb]{0.44,0.44,0.44}{##1}}}
\@namedef{PY@tok@na}{\def\PY@tc##1{\textcolor[rgb]{0.41,0.47,0.13}{##1}}}
\@namedef{PY@tok@nt}{\let\PY@bf=\textbf\def\PY@tc##1{\textcolor[rgb]{0.00,0.50,0.00}{##1}}}
\@namedef{PY@tok@nd}{\def\PY@tc##1{\textcolor[rgb]{0.67,0.13,1.00}{##1}}}
\@namedef{PY@tok@s}{\def\PY@tc##1{\textcolor[rgb]{0.73,0.13,0.13}{##1}}}
\@namedef{PY@tok@sd}{\let\PY@it=\textit\def\PY@tc##1{\textcolor[rgb]{0.73,0.13,0.13}{##1}}}
\@namedef{PY@tok@si}{\let\PY@bf=\textbf\def\PY@tc##1{\textcolor[rgb]{0.64,0.35,0.47}{##1}}}
\@namedef{PY@tok@se}{\let\PY@bf=\textbf\def\PY@tc##1{\textcolor[rgb]{0.67,0.36,0.12}{##1}}}
\@namedef{PY@tok@sr}{\def\PY@tc##1{\textcolor[rgb]{0.64,0.35,0.47}{##1}}}
\@namedef{PY@tok@ss}{\def\PY@tc##1{\textcolor[rgb]{0.10,0.09,0.49}{##1}}}
\@namedef{PY@tok@sx}{\def\PY@tc##1{\textcolor[rgb]{0.00,0.50,0.00}{##1}}}
\@namedef{PY@tok@m}{\def\PY@tc##1{\textcolor[rgb]{0.40,0.40,0.40}{##1}}}
\@namedef{PY@tok@gh}{\let\PY@bf=\textbf\def\PY@tc##1{\textcolor[rgb]{0.00,0.00,0.50}{##1}}}
\@namedef{PY@tok@gu}{\let\PY@bf=\textbf\def\PY@tc##1{\textcolor[rgb]{0.50,0.00,0.50}{##1}}}
\@namedef{PY@tok@gd}{\def\PY@tc##1{\textcolor[rgb]{0.63,0.00,0.00}{##1}}}
\@namedef{PY@tok@gi}{\def\PY@tc##1{\textcolor[rgb]{0.00,0.52,0.00}{##1}}}
\@namedef{PY@tok@gr}{\def\PY@tc##1{\textcolor[rgb]{0.89,0.00,0.00}{##1}}}
\@namedef{PY@tok@ge}{\let\PY@it=\textit}
\@namedef{PY@tok@gs}{\let\PY@bf=\textbf}
\@namedef{PY@tok@ges}{\let\PY@bf=\textbf\let\PY@it=\textit}
\@namedef{PY@tok@gp}{\let\PY@bf=\textbf\def\PY@tc##1{\textcolor[rgb]{0.00,0.00,0.50}{##1}}}
\@namedef{PY@tok@go}{\def\PY@tc##1{\textcolor[rgb]{0.44,0.44,0.44}{##1}}}
\@namedef{PY@tok@gt}{\def\PY@tc##1{\textcolor[rgb]{0.00,0.27,0.87}{##1}}}
\@namedef{PY@tok@err}{\def\PY@bc##1{{\setlength{\fboxsep}{\string -\fboxrule}\fcolorbox[rgb]{1.00,0.00,0.00}{1,1,1}{\strut ##1}}}}
\@namedef{PY@tok@kc}{\let\PY@bf=\textbf\def\PY@tc##1{\textcolor[rgb]{0.00,0.50,0.00}{##1}}}
\@namedef{PY@tok@kd}{\let\PY@bf=\textbf\def\PY@tc##1{\textcolor[rgb]{0.00,0.50,0.00}{##1}}}
\@namedef{PY@tok@kn}{\let\PY@bf=\textbf\def\PY@tc##1{\textcolor[rgb]{0.00,0.50,0.00}{##1}}}
\@namedef{PY@tok@kr}{\let\PY@bf=\textbf\def\PY@tc##1{\textcolor[rgb]{0.00,0.50,0.00}{##1}}}
\@namedef{PY@tok@bp}{\def\PY@tc##1{\textcolor[rgb]{0.00,0.50,0.00}{##1}}}
\@namedef{PY@tok@fm}{\def\PY@tc##1{\textcolor[rgb]{0.00,0.00,1.00}{##1}}}
\@namedef{PY@tok@vc}{\def\PY@tc##1{\textcolor[rgb]{0.10,0.09,0.49}{##1}}}
\@namedef{PY@tok@vg}{\def\PY@tc##1{\textcolor[rgb]{0.10,0.09,0.49}{##1}}}
\@namedef{PY@tok@vi}{\def\PY@tc##1{\textcolor[rgb]{0.10,0.09,0.49}{##1}}}
\@namedef{PY@tok@vm}{\def\PY@tc##1{\textcolor[rgb]{0.10,0.09,0.49}{##1}}}
\@namedef{PY@tok@sa}{\def\PY@tc##1{\textcolor[rgb]{0.73,0.13,0.13}{##1}}}
\@namedef{PY@tok@sb}{\def\PY@tc##1{\textcolor[rgb]{0.73,0.13,0.13}{##1}}}
\@namedef{PY@tok@sc}{\def\PY@tc##1{\textcolor[rgb]{0.73,0.13,0.13}{##1}}}
\@namedef{PY@tok@dl}{\def\PY@tc##1{\textcolor[rgb]{0.73,0.13,0.13}{##1}}}
\@namedef{PY@tok@s2}{\def\PY@tc##1{\textcolor[rgb]{0.73,0.13,0.13}{##1}}}
\@namedef{PY@tok@sh}{\def\PY@tc##1{\textcolor[rgb]{0.73,0.13,0.13}{##1}}}
\@namedef{PY@tok@s1}{\def\PY@tc##1{\textcolor[rgb]{0.73,0.13,0.13}{##1}}}
\@namedef{PY@tok@mb}{\def\PY@tc##1{\textcolor[rgb]{0.40,0.40,0.40}{##1}}}
\@namedef{PY@tok@mf}{\def\PY@tc##1{\textcolor[rgb]{0.40,0.40,0.40}{##1}}}
\@namedef{PY@tok@mh}{\def\PY@tc##1{\textcolor[rgb]{0.40,0.40,0.40}{##1}}}
\@namedef{PY@tok@mi}{\def\PY@tc##1{\textcolor[rgb]{0.40,0.40,0.40}{##1}}}
\@namedef{PY@tok@il}{\def\PY@tc##1{\textcolor[rgb]{0.40,0.40,0.40}{##1}}}
\@namedef{PY@tok@mo}{\def\PY@tc##1{\textcolor[rgb]{0.40,0.40,0.40}{##1}}}
\@namedef{PY@tok@ch}{\let\PY@it=\textit\def\PY@tc##1{\textcolor[rgb]{0.24,0.48,0.48}{##1}}}
\@namedef{PY@tok@cm}{\let\PY@it=\textit\def\PY@tc##1{\textcolor[rgb]{0.24,0.48,0.48}{##1}}}
\@namedef{PY@tok@cpf}{\let\PY@it=\textit\def\PY@tc##1{\textcolor[rgb]{0.24,0.48,0.48}{##1}}}
\@namedef{PY@tok@c1}{\let\PY@it=\textit\def\PY@tc##1{\textcolor[rgb]{0.24,0.48,0.48}{##1}}}
\@namedef{PY@tok@cs}{\let\PY@it=\textit\def\PY@tc##1{\textcolor[rgb]{0.24,0.48,0.48}{##1}}}

\def\PYZbs{\char`\\}
\def\PYZus{\char`\_}
\def\PYZob{\char`\{}
\def\PYZcb{\char`\}}
\def\PYZca{\char`\^}
\def\PYZam{\char`\&}
\def\PYZlt{\char`\<}
\def\PYZgt{\char`\>}
\def\PYZsh{\char`\#}
\def\PYZpc{\char`\%}
\def\PYZdl{\char`\$}
\def\PYZhy{\char`\-}
\def\PYZsq{\char`\'}
\def\PYZdq{\char`\"}
\def\PYZti{\char`\~}
% for compatibility with earlier versions
\def\PYZat{@}
\def\PYZlb{[}
\def\PYZrb{]}
\makeatother


    % For linebreaks inside Verbatim environment from package fancyvrb.
    \makeatletter
        \newbox\Wrappedcontinuationbox
        \newbox\Wrappedvisiblespacebox
        \newcommand*\Wrappedvisiblespace {\textcolor{red}{\textvisiblespace}}
        \newcommand*\Wrappedcontinuationsymbol {\textcolor{red}{\llap{\tiny$\m@th\hookrightarrow$}}}
        \newcommand*\Wrappedcontinuationindent {3ex }
        \newcommand*\Wrappedafterbreak {\kern\Wrappedcontinuationindent\copy\Wrappedcontinuationbox}
        % Take advantage of the already applied Pygments mark-up to insert
        % potential linebreaks for TeX processing.
        %        {, <, #, %, $, ' and ": go to next line.
        %        _, }, ^, &, >, - and ~: stay at end of broken line.
        % Use of \textquotesingle for straight quote.
        \newcommand*\Wrappedbreaksatspecials {%
            \def\PYGZus{\discretionary{\char`\_}{\Wrappedafterbreak}{\char`\_}}%
            \def\PYGZob{\discretionary{}{\Wrappedafterbreak\char`\{}{\char`\{}}%
            \def\PYGZcb{\discretionary{\char`\}}{\Wrappedafterbreak}{\char`\}}}%
            \def\PYGZca{\discretionary{\char`\^}{\Wrappedafterbreak}{\char`\^}}%
            \def\PYGZam{\discretionary{\char`\&}{\Wrappedafterbreak}{\char`\&}}%
            \def\PYGZlt{\discretionary{}{\Wrappedafterbreak\char`\<}{\char`\<}}%
            \def\PYGZgt{\discretionary{\char`\>}{\Wrappedafterbreak}{\char`\>}}%
            \def\PYGZsh{\discretionary{}{\Wrappedafterbreak\char`\#}{\char`\#}}%
            \def\PYGZpc{\discretionary{}{\Wrappedafterbreak\char`\%}{\char`\%}}%
            \def\PYGZdl{\discretionary{}{\Wrappedafterbreak\char`\$}{\char`\$}}%
            \def\PYGZhy{\discretionary{\char`\-}{\Wrappedafterbreak}{\char`\-}}%
            \def\PYGZsq{\discretionary{}{\Wrappedafterbreak\textquotesingle}{\textquotesingle}}%
            \def\PYGZdq{\discretionary{}{\Wrappedafterbreak\char`\"}{\char`\"}}%
            \def\PYGZti{\discretionary{\char`\~}{\Wrappedafterbreak}{\char`\~}}%
        }
        % Some characters . , ; ? ! / are not pygmentized.
        % This macro makes them "active" and they will insert potential linebreaks
        \newcommand*\Wrappedbreaksatpunct {%
            \lccode`\~`\.\lowercase{\def~}{\discretionary{\hbox{\char`\.}}{\Wrappedafterbreak}{\hbox{\char`\.}}}%
            \lccode`\~`\,\lowercase{\def~}{\discretionary{\hbox{\char`\,}}{\Wrappedafterbreak}{\hbox{\char`\,}}}%
            \lccode`\~`\;\lowercase{\def~}{\discretionary{\hbox{\char`\;}}{\Wrappedafterbreak}{\hbox{\char`\;}}}%
            \lccode`\~`\:\lowercase{\def~}{\discretionary{\hbox{\char`\:}}{\Wrappedafterbreak}{\hbox{\char`\:}}}%
            \lccode`\~`\?\lowercase{\def~}{\discretionary{\hbox{\char`\?}}{\Wrappedafterbreak}{\hbox{\char`\?}}}%
            \lccode`\~`\!\lowercase{\def~}{\discretionary{\hbox{\char`\!}}{\Wrappedafterbreak}{\hbox{\char`\!}}}%
            \lccode`\~`\/\lowercase{\def~}{\discretionary{\hbox{\char`\/}}{\Wrappedafterbreak}{\hbox{\char`\/}}}%
            \catcode`\.\active
            \catcode`\,\active
            \catcode`\;\active
            \catcode`\:\active
            \catcode`\?\active
            \catcode`\!\active
            \catcode`\/\active
            \lccode`\~`\~
        }
    \makeatother

    \let\OriginalVerbatim=\Verbatim
    \makeatletter
    \renewcommand{\Verbatim}[1][1]{%
        %\parskip\z@skip
        \sbox\Wrappedcontinuationbox {\Wrappedcontinuationsymbol}%
        \sbox\Wrappedvisiblespacebox {\FV@SetupFont\Wrappedvisiblespace}%
        \def\FancyVerbFormatLine ##1{\hsize\linewidth
            \vtop{\raggedright\hyphenpenalty\z@\exhyphenpenalty\z@
                \doublehyphendemerits\z@\finalhyphendemerits\z@
                \strut ##1\strut}%
        }%
        % If the linebreak is at a space, the latter will be displayed as visible
        % space at end of first line, and a continuation symbol starts next line.
        % Stretch/shrink are however usually zero for typewriter font.
        \def\FV@Space {%
            \nobreak\hskip\z@ plus\fontdimen3\font minus\fontdimen4\font
            \discretionary{\copy\Wrappedvisiblespacebox}{\Wrappedafterbreak}
            {\kern\fontdimen2\font}%
        }%

        % Allow breaks at special characters using \PYG... macros.
        \Wrappedbreaksatspecials
        % Breaks at punctuation characters . , ; ? ! and / need catcode=\active
        \OriginalVerbatim[#1,codes*=\Wrappedbreaksatpunct]%
    }
    \makeatother

    % Exact colors from NB
    \definecolor{incolor}{HTML}{303F9F}
    \definecolor{outcolor}{HTML}{D84315}
    \definecolor{cellborder}{HTML}{CFCFCF}
    \definecolor{cellbackground}{HTML}{F7F7F7}

    % prompt
    \makeatletter
    \newcommand{\boxspacing}{\kern\kvtcb@left@rule\kern\kvtcb@boxsep}
    \makeatother
    \newcommand{\prompt}[4]{
        {\ttfamily\llap{{\color{#2}[#3]:\hspace{3pt}#4}}\vspace{-\baselineskip}}
    }
    

    
    % Prevent overflowing lines due to hard-to-break entities
    \sloppy
    % Setup hyperref package
    \hypersetup{
      breaklinks=true,  % so long urls are correctly broken across lines
      colorlinks=true,
      urlcolor=urlcolor,
      linkcolor=linkcolor,
      citecolor=citecolor,
      }
    % Slightly bigger margins than the latex defaults
    
    \geometry{verbose,tmargin=1in,bmargin=1in,lmargin=1in,rmargin=1in}
    
    

\begin{document}
    
    \maketitle
    
    

    
    \section{MATH3090 Assignment 1}\label{math3090-assignment-1}

Student: Hugo Burton (s4698512) Date Due: Tuesday March 19 @ 1pm

    \begin{tcolorbox}[breakable, size=fbox, boxrule=1pt, pad at break*=1mm,colback=cellbackground, colframe=cellborder]
\prompt{In}{incolor}{1}{\boxspacing}
\begin{Verbatim}[commandchars=\\\{\}]
\PY{k+kn}{import} \PY{n+nn}{math}
\PY{k+kn}{from} \PY{n+nn}{typing} \PY{k+kn}{import} \PY{n}{Dict}
\PY{k+kn}{from} \PY{n+nn}{colorama} \PY{k+kn}{import} \PY{n}{Fore}\PY{p}{,} \PY{n}{Style}
\PY{k+kn}{import} \PY{n+nn}{numpy} \PY{k}{as} \PY{n+nn}{np}
\PY{k+kn}{from} \PY{n+nn}{IPython}\PY{n+nn}{.}\PY{n+nn}{display} \PY{k+kn}{import} \PY{n}{Markdown}\PY{p}{,} \PY{n}{display}

\PY{k+kn}{import} \PY{n+nn}{bond}
\PY{k+kn}{import} \PY{n+nn}{interest}
\PY{k+kn}{import} \PY{n+nn}{newtons}
\PY{k+kn}{import} \PY{n+nn}{display} \PY{k}{as} \PY{n+nn}{dsp}
\end{Verbatim}
\end{tcolorbox}

    \subsection{Question 1 (6 marks)}\label{question-1-6-marks}

\subsubsection{Part A (3 marks)}\label{part-a-3-marks}

Suppose a company issues a zero coupon bond with face value \$10,000 and
which matures in 20 years. Calculate the price given:

    \begin{tcolorbox}[breakable, size=fbox, boxrule=1pt, pad at break*=1mm,colback=cellbackground, colframe=cellborder]
\prompt{In}{incolor}{2}{\boxspacing}
\begin{Verbatim}[commandchars=\\\{\}]
\PY{n}{face\PYZus{}value} \PY{o}{=} \PY{l+m+mi}{10\PYZus{}000}
\PY{n}{years\PYZus{}to\PYZus{}maturity} \PY{o}{=} \PY{l+m+mi}{20}
\end{Verbatim}
\end{tcolorbox}

    \begin{enumerate}
\def\labelenumi{\roman{enumi}.}
\tightlist
\item
  An 8\% discount compound annual yield, compounded annually.
\end{enumerate}

    \begin{tcolorbox}[breakable, size=fbox, boxrule=1pt, pad at break*=1mm,colback=cellbackground, colframe=cellborder]
\prompt{In}{incolor}{3}{\boxspacing}
\begin{Verbatim}[commandchars=\\\{\}]
\PY{c+c1}{\PYZsh{} Question i}
\PY{n}{interest\PYZus{}rate} \PY{o}{=} \PY{l+m+mf}{0.08}
\PY{n}{compounding\PYZus{}frequency\PYZus{}yr} \PY{o}{=} \PY{l+m+mi}{1}

\PY{n}{price\PYZus{}i} \PY{o}{=} \PY{n}{bond}\PY{o}{.}\PY{n}{price\PYZus{}zero\PYZus{}coupon\PYZus{}bond\PYZus{}discrete}\PY{p}{(}
    \PY{n}{face\PYZus{}value}\PY{p}{,} \PY{n}{years\PYZus{}to\PYZus{}maturity}\PY{p}{,} \PY{n}{interest\PYZus{}rate}\PY{p}{,} \PY{n}{compounding\PYZus{}frequency\PYZus{}yr}\PY{p}{)}

\PY{n}{dsp}\PY{o}{.}\PY{n}{display\PYZus{}answer}\PY{p}{(}\PY{n}{price\PYZus{}i}\PY{p}{)}
\end{Verbatim}
\end{tcolorbox}

    \begin{Verbatim}[commandchars=\\\{\}]
\textcolor{ansi-green}{Answer:} \$2145.48
    \end{Verbatim}

    \begin{enumerate}
\def\labelenumi{\roman{enumi}.}
\setcounter{enumi}{1}
\tightlist
\item
  An 8\% discount continuous annual yield, compounded semi-annually.
\end{enumerate}

    \begin{tcolorbox}[breakable, size=fbox, boxrule=1pt, pad at break*=1mm,colback=cellbackground, colframe=cellborder]
\prompt{In}{incolor}{4}{\boxspacing}
\begin{Verbatim}[commandchars=\\\{\}]
\PY{c+c1}{\PYZsh{} Question ii}
\PY{n}{interest\PYZus{}rate} \PY{o}{=} \PY{l+m+mf}{0.08}

\PY{n}{price\PYZus{}ii} \PY{o}{=} \PY{n}{bond}\PY{o}{.}\PY{n}{price\PYZus{}zero\PYZus{}coupon\PYZus{}bond\PYZus{}continuous}\PY{p}{(}
    \PY{n}{face\PYZus{}value}\PY{p}{,} \PY{n}{years\PYZus{}to\PYZus{}maturity}\PY{p}{,} \PY{n}{interest\PYZus{}rate}\PY{p}{)}

\PY{n}{dsp}\PY{o}{.}\PY{n}{display\PYZus{}answer}\PY{p}{(}\PY{n}{price\PYZus{}ii}\PY{p}{)}
\end{Verbatim}
\end{tcolorbox}

    \begin{Verbatim}[commandchars=\\\{\}]
\textcolor{ansi-green}{Answer:} \$2018.97
    \end{Verbatim}

    \begin{enumerate}
\def\labelenumi{\roman{enumi}.}
\setcounter{enumi}{2}
\tightlist
\item
  A nonconstant yield of \(y(t) = 0.06 + 0.2 t e^{-t^2}\).
\end{enumerate}

    \begin{tcolorbox}[breakable, size=fbox, boxrule=1pt, pad at break*=1mm,colback=cellbackground, colframe=cellborder]
\prompt{In}{incolor}{5}{\boxspacing}
\begin{Verbatim}[commandchars=\\\{\}]
\PY{c+c1}{\PYZsh{} Question iii}
\PY{n}{q} \PY{o}{=} \PY{l+s+s2}{\PYZdq{}}\PY{l+s+s2}{iii}\PY{l+s+s2}{\PYZdq{}}
\PY{k}{def} \PY{n+nf}{yield\PYZus{}function}\PY{p}{(}\PY{n}{t}\PY{p}{)}\PY{p}{:} \PY{k}{return} \PY{l+m+mf}{0.06} \PY{o}{+} \PY{l+m+mf}{0.2} \PY{o}{*} \PY{n}{t} \PY{o}{*} \PY{n}{math}\PY{o}{.}\PY{n}{exp}\PY{p}{(}\PY{o}{\PYZhy{}}\PY{n}{t}\PY{o}{*}\PY{o}{*}\PY{l+m+mi}{2}\PY{p}{)}

\PY{n}{price\PYZus{}iii} \PY{o}{=} \PY{n}{bond}\PY{o}{.}\PY{n}{price\PYZus{}zero\PYZus{}coupon\PYZus{}bond\PYZus{}nonconstant\PYZus{}yield}\PY{p}{(}
    \PY{n}{face\PYZus{}value}\PY{p}{,} \PY{n}{years\PYZus{}to\PYZus{}maturity}\PY{p}{,} \PY{n}{yield\PYZus{}function}\PY{p}{)}

\PY{n}{dsp}\PY{o}{.}\PY{n}{display\PYZus{}answer}\PY{p}{(}\PY{n}{price\PYZus{}iii}\PY{p}{)}
\end{Verbatim}
\end{tcolorbox}

    \begin{Verbatim}[commandchars=\\\{\}]
\textcolor{ansi-green}{Answer:} \$2725.32
    \end{Verbatim}

    \subsubsection{Part B (3 marks)}\label{part-b-3-marks}

A 10 year \$10,000 government bond has a coupon rate of 5\% payable
quarterly and yields 7\%. Calculate the price.

    \begin{tcolorbox}[breakable, size=fbox, boxrule=1pt, pad at break*=1mm,colback=cellbackground, colframe=cellborder]
\prompt{In}{incolor}{6}{\boxspacing}
\begin{Verbatim}[commandchars=\\\{\}]
\PY{c+c1}{\PYZsh{} Part b (3 marks)}
\PY{n}{face\PYZus{}value} \PY{o}{=} \PY{l+m+mi}{10\PYZus{}000}
\PY{n}{years\PYZus{}to\PYZus{}maturity} \PY{o}{=} \PY{l+m+mi}{10}
\PY{n}{coupon\PYZus{}rate} \PY{o}{=} \PY{l+m+mf}{0.05}
\PY{n}{interest\PYZus{}rate} \PY{o}{=} \PY{l+m+mf}{0.07}
\PY{n}{compounding\PYZus{}frequency\PYZus{}yr} \PY{o}{=} \PY{l+m+mi}{4}

\PY{n}{price\PYZus{}b} \PY{o}{=} \PY{n}{bond}\PY{o}{.}\PY{n}{price\PYZus{}coupon\PYZus{}bearing\PYZus{}bond\PYZus{}discrete}\PY{p}{(}
    \PY{n}{face\PYZus{}value}\PY{p}{,} \PY{n}{years\PYZus{}to\PYZus{}maturity}\PY{p}{,} \PY{n}{coupon\PYZus{}rate}\PY{p}{,} \PY{n}{interest\PYZus{}rate}\PY{p}{,} \PY{n}{compounding\PYZus{}frequency\PYZus{}yr}\PY{p}{)}

\PY{n}{dsp}\PY{o}{.}\PY{n}{display\PYZus{}answer}\PY{p}{(}\PY{n}{price\PYZus{}b}\PY{p}{)}
\end{Verbatim}
\end{tcolorbox}

    \begin{Verbatim}[commandchars=\\\{\}]
\textcolor{ansi-green}{Answer:} \$8570.29
    \end{Verbatim}

    \subsection{Question 2 (6 marks)}\label{question-2-6-marks}

Consider the cash flow

\[C_0 = -3x,\hspace{0.5cm} C_1 = 5,\hspace{0.5cm} C_2 = x\]

(at periods 0, 1, 2 respectively) for some \(x > 0\).

    \subsection{Part A (3 marks)}\label{part-a-3-marks}

Apply the discount process \(d(k) = (1 + r)^{-k}\) so that the present
value is

\[P = \sum_{k=0}^{2} d(k) C_k\]

What is the range of \(x\) such that \(P > 0\) when \(r = 5\%\)?

    \begin{align}
    0 &< P \\
    0 &< \sum_{i=0}^{2} d(k) C_k \\
    0 &< \left[(1 + 0.05)^{-0} \cdot -3x\right] + \left[(1 + 0.05)^{-1} \cdot 5\right] + \left[(1 + 0.05)^{-2} \cdot x\right] \\
    0 &< \left[1 \cdot -3x\right] + \left[\frac{5}{1 + 0.05}\right] + \left[\frac{x}{(1 + 0.05)^{2}}\right] \\
    0 &< -3x + \frac{5}{1.05} + \frac{x}{1.05^{2}} \\
    0 &< -3 \cdot 1.05^2 x + 5 \cdot 1.05 + x \\
    0 &< -2.3075 x + 5.25 \\
    2.3075 x &< 5.25 \\
    x &< \frac{5.25}{2.3075} \\
    x &\lessapprox 2.275 \\
\end{align}

Therefore, when \(r = 5\%\), \(P > 0\) holds when
\(x \lessapprox 2.275\).

We can verify this in code numerically as follows.

    \begin{tcolorbox}[breakable, size=fbox, boxrule=1pt, pad at break*=1mm,colback=cellbackground, colframe=cellborder]
\prompt{In}{incolor}{7}{\boxspacing}
\begin{Verbatim}[commandchars=\\\{\}]
\PY{n}{r} \PY{o}{=} \PY{l+m+mf}{0.05}
\PY{n}{cash\PYZus{}flow} \PY{o}{=} \PY{k}{lambda} \PY{n}{k}\PY{p}{,} \PY{n}{x}\PY{p}{:} \PY{o}{\PYZhy{}}\PY{l+m+mi}{3}\PY{o}{*}\PY{n}{x} \PY{k}{if} \PY{n}{k} \PY{o}{==} \PY{l+m+mi}{0} \PY{k}{else} \PY{p}{(}\PY{l+m+mi}{5} \PY{k}{if} \PY{n}{k} \PY{o}{==} \PY{l+m+mi}{1} \PY{k}{else} \PY{n}{x}\PY{p}{)}

\PY{n}{accept\PYZus{}condition} \PY{o}{=} \PY{k}{lambda} \PY{n}{p}\PY{p}{:} \PY{n}{p} \PY{o}{\PYZgt{}} \PY{l+m+mi}{0}

\PY{c+c1}{\PYZsh{} Function to calculate present value based on some value of x}
\PY{k}{def} \PY{n+nf}{present\PYZus{}value}\PY{p}{(}\PY{n}{x}\PY{p}{:} \PY{n+nb}{float}\PY{p}{)} \PY{o}{\PYZhy{}}\PY{o}{\PYZgt{}} \PY{n+nb}{float}\PY{p}{:}
    \PY{n}{present\PYZus{}value} \PY{o}{=} \PY{n+nb}{sum}\PY{p}{(}\PY{n}{cash\PYZus{}flow}\PY{p}{(}\PY{n}{k}\PY{p}{,} \PY{n}{x}\PY{p}{)} \PY{o}{*} \PY{n}{interest}\PY{o}{.}\PY{n}{discrete\PYZus{}compound\PYZus{}interest\PYZus{}discounted}\PY{p}{(}\PY{n}{r}\PY{p}{,} \PY{n}{k}\PY{p}{,} \PY{l+m+mi}{1}\PY{p}{)} \PYZbs{}
                        \PY{k}{for} \PY{n}{k} \PY{o+ow}{in} \PY{n+nb}{range}\PY{p}{(}\PY{l+m+mi}{2}\PY{o}{+}\PY{l+m+mi}{1}\PY{p}{)}\PY{p}{)}
    
    \PY{k}{return} \PY{n}{present\PYZus{}value}

\PY{n}{cash\PYZus{}flows\PYZus{}x}\PY{p}{:} \PY{n}{Dict}\PY{p}{[}\PY{n+nb}{float}\PY{p}{,} \PY{n+nb}{float}\PY{p}{]} \PY{o}{=} \PY{p}{\PYZob{}}\PY{p}{\PYZcb{}}
\PY{n}{x\PYZus{}min} \PY{o}{=} \PY{o}{\PYZhy{}}\PY{l+m+mf}{100.0}
\PY{n}{x\PYZus{}max} \PY{o}{=} \PY{l+m+mf}{100.0}
\PY{n}{x\PYZus{}step} \PY{o}{=} \PY{l+m+mf}{0.01}
\PY{n}{accept\PYZus{}min\PYZus{}x} \PY{o}{=} \PY{l+m+mi}{0}
\PY{n}{accept\PYZus{}max\PYZus{}x} \PY{o}{=} \PY{k+kc}{None}

\PY{c+c1}{\PYZsh{} Loop over a wide range of x}
\PY{k}{for} \PY{n}{x} \PY{o+ow}{in} \PY{n}{np}\PY{o}{.}\PY{n}{arange}\PY{p}{(}\PY{n}{x\PYZus{}min}\PY{p}{,} \PY{n}{x\PYZus{}max}\PY{p}{,} \PY{n}{x\PYZus{}step}\PY{p}{)}\PY{p}{:}
    \PY{n}{cash\PYZus{}flow\PYZus{}x} \PY{o}{=} \PY{n}{present\PYZus{}value}\PY{p}{(}\PY{n}{x}\PY{p}{)}

    \PY{n}{cash\PYZus{}flows\PYZus{}x}\PY{p}{[}\PY{n}{x}\PY{p}{]} \PY{o}{=} \PY{n}{cash\PYZus{}flow\PYZus{}x}

    \PY{k}{if} \PY{n}{accept\PYZus{}condition}\PY{p}{(}\PY{n}{cash\PYZus{}flow\PYZus{}x}\PY{p}{)}\PY{p}{:}
        \PY{c+c1}{\PYZsh{} Min}
        \PY{k}{if} \PY{o+ow}{not} \PY{n}{accept\PYZus{}min\PYZus{}x}\PY{p}{:}
            \PY{n}{accept\PYZus{}min\PYZus{}x} \PY{o}{=} \PY{n}{x}
        \PY{k}{elif} \PY{n}{x} \PY{o}{\PYZlt{}} \PY{n}{accept\PYZus{}min\PYZus{}x}\PY{p}{:}
            \PY{n}{accept\PYZus{}min\PYZus{}x} \PY{o}{=} \PY{n}{x}

        \PY{c+c1}{\PYZsh{} Max}
        \PY{k}{if} \PY{o+ow}{not} \PY{n}{accept\PYZus{}max\PYZus{}x}\PY{p}{:}
            \PY{n}{accept\PYZus{}max\PYZus{}x} \PY{o}{=} \PY{n}{x}
        \PY{k}{elif} \PY{n}{x} \PY{o}{\PYZgt{}} \PY{n}{accept\PYZus{}max\PYZus{}x}\PY{p}{:}
            \PY{n}{accept\PYZus{}max\PYZus{}x} \PY{o}{=} \PY{n}{x}

\PY{k}{if} \PY{n}{accept\PYZus{}min\PYZus{}x} \PY{o}{==} \PY{n}{x\PYZus{}min}\PY{p}{:}
    \PY{n}{accept\PYZus{}min\PYZus{}x} \PY{o}{=} \PY{l+m+mi}{0}

\PY{k}{if} \PY{n}{accept\PYZus{}max\PYZus{}x} \PY{o}{==} \PY{n}{x\PYZus{}max}\PY{p}{:}
    \PY{n}{accept\PYZus{}max\PYZus{}x} \PY{o}{=} \PY{n}{math}\PY{o}{.}\PY{n}{inf}

\PY{n+nb}{print}\PY{p}{(}
    \PY{l+s+sa}{f}\PY{l+s+s2}{\PYZdq{}}\PY{l+s+s2}{Range of x such that P \PYZgt{} 0 when r = }\PY{l+s+si}{\PYZob{}}\PY{n}{r}\PY{+w}{ }\PY{o}{*}\PY{+w}{ }\PY{l+m+mi}{100}\PY{l+s+si}{\PYZcb{}}\PY{l+s+s2}{\PYZpc{}: }\PY{l+s+si}{\PYZob{}}\PY{n}{accept\PYZus{}min\PYZus{}x}\PY{l+s+si}{\PYZcb{}}\PY{l+s+s2}{ \PYZlt{} x \PYZlt{} }\PY{l+s+si}{\PYZob{}}\PY{n}{accept\PYZus{}max\PYZus{}x}\PY{l+s+si}{\PYZcb{}}\PY{l+s+s2}{\PYZdq{}}\PY{p}{)}
\end{Verbatim}
\end{tcolorbox}

    \begin{Verbatim}[commandchars=\\\{\}]
Range of x such that P > 0 when r = 5.0\%: 0 < x < 2.2700000000523204
    \end{Verbatim}

    \subsection{Part B (3 marks)}\label{part-b-3-marks}

The IRR (internal rate of return) is \(r\) such that \(P = 0\). For what
range of \(x\) will there be a unique, strictly positive IRR?

    \begin{align}
    P &= \sum_{k=0}^{2} d(k) \cdot C_k \\
    0 &= \left[d(0) \cdot -3x\right] + \left[d(1) \cdot 5\right] + \left[d(2) \cdot x\right] \\
    &= -3x (1 + r)^{-0} + 5 (1 + r)^{-1} + x (1 + r)^{-2} \\
    &= -3x \cdot 1 + \frac{5}{1 + r} + \frac{x}{(1 + r)^2} \\
    &= -3x (1 + r)^2 + 5 (1 + r) + x \\
    &= -3x (1^2 + 2r + r^2) + 5 + 5r + x \\
    &= -3x - 6xr - 3xr^2 + 5 + 5r + x \\
    &= -3xr^2 + (5 - 6x)r + (x - 3x + 5) \\
    &= -3xr^2 + (5 - 6x)r + (5 - 2x)
\end{align}

As we want \(r > 0\), then solve for \(\Delta > 0\)

\begin{align}
    0 &< \Delta \\
    0 &< (5 - 6x)^2 - 4 \cdot (-3x) \cdot (5 - 2x) \\
    0 &< (25 - 60x + 36x^2) + 12x \cdot (5 - 2x) \\
    0 &< 36x^2 - 60x + 25 + 60x - 24x^2 \\
    0 &< 12x^2 + 25
\end{align}

This will always hold \(\forall x \in \mathbb{R}^+\)

Therefore, if \(x > 0\), then IRR \(= \left\{r | P = 0\right\} > 0\).

    \subsection{Question 3 (8 marks)}\label{question-3-8-marks}

\begin{longtable}[]{@{}cc@{}}
\toprule\noalign{}
Cashflows (\(C_i\)) & Times (\(t_i\)) \\
\midrule\noalign{}
\endhead
\bottomrule\noalign{}
\endlastfoot
2.3 & 1.0 \\
2.9 & 2.0 \\
3.0 & 3.0 \\
3.2 & 4.0 \\
4.0 & 5.0 \\
3.8 & 6.0 \\
4.2 & 7.0 \\
4.8 & 8.0 \\
5.5 & 9.0 \\
105 & 10.0 \\
\end{longtable}

Table 1: Bond Cashflows

In this question, consider a bond with a set of cashflows given in Table
1. Here, note that the face value \(F\) is already included in the last
cashflow. Let \(y\) be the yield to maturity, \(t_i\) be the time of the
\(i^{\text{th}}\) cashflow \(C_i\), and \(PV = 100\) be the market price
of the bond at \(t = 0\). Assume continuous compounding. Then \(y\)
solves

\[PV = \sum_{i} C_i e^{-y t_i}\]

    \subsection{Part A (3 marks)}\label{part-a-3-marks}

Write out the Newton iteration to compute \(y_{n+1}\) from \(y_n\) (see
L2.49). Specifically, clearly indicate the functions \(f(y)\) and
\(f'(y)\).

    In this question

\[f(y) = \left[\sum_{t=1}^{10} C_t \cdot \beta(y, t)\right] - P = \left[\sum_{t=1}^{10}C_t \exp\left\{-y \cdot t\right\}\right] - P\]

and

\[f'(y) = \sum_{t=1}^{10} C_t \cdot \beta'(y, t) = \sum_{t=1}^{10} C_t \cdot \frac{d}{dx}\exp\left\{-y \cdot t\right\} = - \sum_{t=1}^{10}t C_t \exp\left\{-y \cdot t\right\}\]

Using these definitions, perform the following

\begin{enumerate}
\def\labelenumi{\arabic{enumi}.}
\tightlist
\item
  Choose an intial value of \(x_0\), say \(x_0 = 0.05\)
\item
  Compute the following until the termination condition
\end{enumerate}

\begin{align}
    x_{n+1} &\approx x_n - \frac{f(x_n)}{f'(x_n)} \\
    &\approx x_n + \frac{\left[\sum_{t=1}^{10}C_t \exp\left\{-x_n \cdot t\right\}\right] - P}{- \sum_{t=1}^{10}t C_t \exp\left\{-x_n \cdot t\right\}}
\end{align}

\begin{enumerate}
\def\labelenumi{\arabic{enumi}.}
\setcounter{enumi}{2}
\tightlist
\item
  Terminate when \(|x_{n+1} - x_{n}| < \epsilon\) or
  \(|f(x_{n+1})| < \epsilon\)
\end{enumerate}

    \subsection{Part B (5 marks)}\label{part-b-5-marks}

Implement the above Newton iteration in Matlab (I'm using Python) using
the stopping criteria

\[|y_{n+1} - y_n| < 10^{-8}.\]

Fill in Table 2 for \(y_0 = 0.05\) (add rows as necessary).

In addition, try with larger values for \(y_0\) and observe the accuracy
and convergence speed. How does the performance change?

    \begin{tcolorbox}[breakable, size=fbox, boxrule=1pt, pad at break*=1mm,colback=cellbackground, colframe=cellborder]
\prompt{In}{incolor}{8}{\boxspacing}
\begin{Verbatim}[commandchars=\\\{\}]
\PY{c+c1}{\PYZsh{} Part b (5 marks)}
\PY{n}{cashflows} \PY{o}{=} \PY{p}{[}\PY{l+m+mf}{2.3}\PY{p}{,} \PY{l+m+mf}{2.9}\PY{p}{,} \PY{l+m+mf}{3.0}\PY{p}{,} \PY{l+m+mf}{3.2}\PY{p}{,} \PY{l+m+mf}{4.0}\PY{p}{,} \PY{l+m+mf}{3.8}\PY{p}{,} \PY{l+m+mf}{4.2}\PY{p}{,} \PY{l+m+mf}{4.8}\PY{p}{,} \PY{l+m+mf}{5.5}\PY{p}{,} \PY{l+m+mi}{105}\PY{p}{]}

\PY{c+c1}{\PYZsh{} Market price of the bond at t = 0}
\PY{n}{PV} \PY{o}{=} \PY{l+m+mi}{100}

\PY{k}{def} \PY{n+nf}{f}\PY{p}{(}\PY{n}{y}\PY{p}{)}\PY{p}{:} \PY{k}{return} \PY{n+nb}{sum}\PY{p}{(}
    \PY{n}{cashflows}\PY{p}{[}\PY{n}{t}\PY{o}{\PYZhy{}}\PY{l+m+mi}{1}\PY{p}{]} \PY{o}{*} \PY{n}{interest}\PY{o}{.}\PY{n}{continuous\PYZus{}compound\PYZus{}interest\PYZus{}discounted}\PY{p}{(}\PY{n}{y}\PY{p}{,} \PY{n}{t}\PY{p}{)} \PY{k}{for} \PY{n}{t} \PY{o+ow}{in} \PY{n+nb}{range}\PY{p}{(}\PY{l+m+mi}{1}\PY{p}{,} \PY{n+nb}{len}\PY{p}{(}\PY{n}{cashflows}\PY{p}{)}\PY{o}{+}\PY{l+m+mi}{1}\PY{p}{)}\PY{p}{)} \PY{o}{\PYZhy{}} \PY{n}{PV}

\PY{k}{def} \PY{n+nf}{f\PYZus{}prime}\PY{p}{(}\PY{n}{y}\PY{p}{)}\PY{p}{:} \PY{k}{return} \PY{o}{\PYZhy{}} \PYZbs{}
    \PY{n+nb}{sum}\PY{p}{(}\PY{p}{(}\PY{p}{(}\PY{n}{t} \PY{o}{*} \PY{n}{cashflows}\PY{p}{[}\PY{n}{t}\PY{p}{]}\PY{p}{)} \PY{o}{/} \PY{p}{(}\PY{p}{(}\PY{l+m+mi}{1} \PY{o}{+} \PY{n}{y}\PY{p}{)}\PY{o}{*}\PY{o}{*}\PY{p}{(}\PY{n}{t} \PY{o}{+} \PY{l+m+mi}{1}\PY{p}{)}\PY{p}{)}\PY{p}{)}
        \PY{k}{for} \PY{n}{t} \PY{o+ow}{in} \PY{n+nb}{range}\PY{p}{(}\PY{n+nb}{len}\PY{p}{(}\PY{n}{cashflows}\PY{p}{)}\PY{p}{)}\PY{p}{)}

    
\PY{n}{eps} \PY{o}{=} \PY{l+m+mf}{1e\PYZhy{}8}

\PY{c+c1}{\PYZsh{} Set initial y value}
\PY{n}{x\PYZus{}0} \PY{o}{=} \PY{l+m+mf}{0.05}

\PY{c+c1}{\PYZsh{} Print table header}
\PY{n}{col\PYZus{}heads} \PY{o}{=} \PY{p}{[}\PY{l+s+s2}{\PYZdq{}}\PY{l+s+s2}{\PYZdl{}\PYZdl{}n\PYZdl{}\PYZdl{}}\PY{l+s+s2}{\PYZdq{}}\PY{p}{,} \PY{l+s+s2}{\PYZdq{}}\PY{l+s+s2}{\PYZdl{}\PYZdl{}y\PYZus{}n\PYZdl{}\PYZdl{}}\PY{l+s+s2}{\PYZdq{}} \PY{p}{,} \PY{l+s+s2}{\PYZdq{}}\PY{l+s+s2}{\PYZdl{}\PYZdl{}|y\PYZus{}}\PY{l+s+si}{\PYZob{}n\PYZcb{}}\PY{l+s+s2}{\PYZhy{}y\PYZus{}}\PY{l+s+s2}{\PYZob{}}\PY{l+s+s2}{n\PYZhy{}1\PYZcb{}|\PYZdl{}\PYZdl{}}\PY{l+s+s2}{\PYZdq{}}\PY{p}{]}
\PY{n}{col\PYZus{}spaces} \PY{o}{=} \PY{p}{[}\PY{l+m+mi}{3}\PY{p}{,} \PY{l+m+mi}{11}\PY{p}{,} \PY{l+m+mi}{17}\PY{p}{]}

\PY{n}{md\PYZus{}table} \PY{o}{=} \PY{l+s+s2}{\PYZdq{}}\PY{l+s+s2}{\PYZdq{}}
\PY{n}{header\PYZus{}row} \PY{o}{=} \PY{l+s+s2}{\PYZdq{}}\PY{l+s+s2}{\PYZdq{}}
\PY{n}{format\PYZus{}row} \PY{o}{=} \PY{l+s+s2}{\PYZdq{}}\PY{l+s+s2}{\PYZdq{}}

\PY{k}{for} \PY{n}{i}\PY{p}{,} \PY{n}{col\PYZus{}head} \PY{o+ow}{in} \PY{n+nb}{enumerate}\PY{p}{(}\PY{n}{col\PYZus{}heads}\PY{p}{)}\PY{p}{:}
    \PY{n}{space} \PY{o}{=} \PY{n}{col\PYZus{}spaces}\PY{p}{[}\PY{n}{i}\PY{p}{]}
    \PY{n}{part} \PY{o}{=} \PY{l+s+sa}{f}\PY{l+s+s2}{\PYZdq{}}\PY{l+s+s2}{|}\PY{l+s+si}{\PYZob{}}\PY{n}{col\PYZus{}head}\PY{l+s+si}{:}\PY{l+s+s2}{\PYZca{}}\PY{l+s+si}{\PYZob{}}\PY{n}{space}\PY{l+s+si}{\PYZcb{}}\PY{l+s+si}{\PYZcb{}}\PY{l+s+s2}{\PYZdq{}}
    \PY{n}{header\PYZus{}row} \PY{o}{+}\PY{o}{=} \PY{n}{part}
    \PY{n}{middle} \PY{o}{=} \PY{l+s+s1}{\PYZsq{}}\PY{l+s+s1}{\PYZhy{}}\PY{l+s+s1}{\PYZsq{}}\PY{o}{*}\PY{p}{(}\PY{n+nb}{max}\PY{p}{(}\PY{l+m+mi}{1}\PY{p}{,} \PY{n+nb}{len}\PY{p}{(}\PY{n}{part}\PY{p}{)} \PY{o}{\PYZhy{}} \PY{l+m+mi}{2} \PY{o}{\PYZhy{}} \PY{l+m+mi}{2}\PY{p}{)}\PY{p}{)}
    \PY{n}{format\PYZus{}row} \PY{o}{+}\PY{o}{=} \PY{l+s+sa}{f}\PY{l+s+s2}{\PYZdq{}}\PY{l+s+s2}{| :}\PY{l+s+si}{\PYZob{}}\PY{n}{middle}\PY{l+s+si}{\PYZcb{}}\PY{l+s+s2}{: }\PY{l+s+s2}{\PYZdq{}}

\PY{n}{header\PYZus{}row} \PY{o}{+}\PY{o}{=} \PY{l+s+s2}{\PYZdq{}}\PY{l+s+s2}{|}\PY{l+s+s2}{\PYZdq{}}
\PY{n}{format\PYZus{}row} \PY{o}{+}\PY{o}{=} \PY{l+s+s2}{\PYZdq{}}\PY{l+s+s2}{|}\PY{l+s+s2}{\PYZdq{}}

\PY{c+c1}{\PYZsh{} Solve y using Newton\PYZsq{}s method given f and PV as inputs}

\PY{n}{approx}\PY{p}{,} \PY{n}{table\PYZus{}rows}\PY{p}{,} \PY{n}{\PYZus{}} \PY{o}{=} \PY{n}{newtons}\PY{o}{.}\PY{n}{newtons\PYZus{}method}\PY{p}{(}\PY{n}{f}\PY{p}{,} \PY{n}{f\PYZus{}prime}\PY{p}{,} \PY{n}{x\PYZus{}0}\PY{p}{,} \PY{n}{eps}\PY{p}{,} \PY{l+m+mi}{9999999}\PY{p}{,} \PY{n}{generate\PYZus{}table}\PY{o}{=}\PY{k+kc}{True}\PY{p}{,} \PY{n}{log}\PY{o}{=}\PY{k+kc}{False}\PY{p}{,} 
                                               \PY{n}{col\PYZus{}spaces}\PY{o}{=}\PY{n}{col\PYZus{}spaces}\PY{p}{,} \PY{n}{precision}\PY{o}{=}\PY{l+m+mi}{10}\PY{p}{)}

\PY{n}{md\PYZus{}table} \PY{o}{+}\PY{o}{=} \PY{n}{header\PYZus{}row} \PY{o}{+} \PY{l+s+s2}{\PYZdq{}}\PY{l+s+s2}{  }\PY{l+s+se}{\PYZbs{}n}\PY{l+s+s2}{\PYZdq{}}
\PY{n}{md\PYZus{}table} \PY{o}{+}\PY{o}{=} \PY{n}{format\PYZus{}row} \PY{o}{+} \PY{l+s+s2}{\PYZdq{}}\PY{l+s+s2}{  }\PY{l+s+se}{\PYZbs{}n}\PY{l+s+s2}{\PYZdq{}}
\PY{n}{md\PYZus{}table} \PY{o}{+}\PY{o}{=} \PY{l+s+s2}{\PYZdq{}}\PY{l+s+s2}{  }\PY{l+s+se}{\PYZbs{}n}\PY{l+s+s2}{\PYZdq{}}\PY{o}{.}\PY{n}{join}\PY{p}{(}\PY{n}{table\PYZus{}rows}\PY{p}{)}

\PY{k}{def} \PY{n+nf}{printmd}\PY{p}{(}\PY{n}{string}\PY{p}{)}\PY{p}{:}
    \PY{n}{display}\PY{p}{(}\PY{n}{Markdown}\PY{p}{(}\PY{n}{string}\PY{p}{)}\PY{p}{)}
    
\PY{n}{printmd}\PY{p}{(}\PY{n}{md\PYZus{}table}\PY{p}{)}
\end{Verbatim}
\end{tcolorbox}

    \begin{longtable}[]{@{}ccc@{}}
\toprule\noalign{}
\(n\) & \(y_n\) & \(|y_{n}-y_{n-1}|\) \\
\midrule\noalign{}
\endhead
\bottomrule\noalign{}
\endlastfoot
0 & 0.0345487861 & 0.0154512139 \\
1 & 0.0372734985 & 0.0027247125 \\
2 & 0.0369704389 & 0.0003030597 \\
3 & 0.0370078067 & 3.73678e-05 \\
4 & 0.0370032489 & 4.5578e-06 \\
5 & 0.0370038056 & 5.567e-07 \\
6 & 0.0370037376 & 6.8e-08 \\
7 & 0.0370037459 & 8.3e-09 \\
\end{longtable}

    
    \subsubsection{\texorpdfstring{Part ii: Larger values of
\(y_0\)}{Part ii: Larger values of y\_0}}\label{part-ii-larger-values-of-y_0}

    \begin{tcolorbox}[breakable, size=fbox, boxrule=1pt, pad at break*=1mm,colback=cellbackground, colframe=cellborder]
\prompt{In}{incolor}{11}{\boxspacing}
\begin{Verbatim}[commandchars=\\\{\}]
\PY{n}{increment} \PY{o}{=} \PY{l+m+mf}{0.01}
\PY{n}{y\PYZus{}0\PYZus{}vals} \PY{o}{=} \PY{p}{[}\PY{n}{x} \PY{k}{for} \PY{n}{x} \PY{o+ow}{in} \PY{n}{np}\PY{o}{.}\PY{n}{arange}\PY{p}{(}\PY{l+m+mf}{0.05}\PY{p}{,} \PY{l+m+mf}{0.25}\PY{o}{+}\PY{n}{increment}\PY{p}{,} \PY{n}{increment}\PY{p}{)}\PY{p}{]}

\PY{k}{for} \PY{n}{y\PYZus{}0} \PY{o+ow}{in} \PY{n}{y\PYZus{}0\PYZus{}vals}\PY{p}{:}
    \PY{n+nb}{print}\PY{p}{(}
        \PY{l+s+sa}{f}\PY{l+s+s2}{\PYZdq{}}\PY{l+s+si}{\PYZob{}}\PY{n}{Fore}\PY{o}{.}\PY{n}{CYAN}\PY{l+s+si}{\PYZcb{}}\PY{l+s+s2}{y\PYZus{}0 = }\PY{l+s+si}{\PYZob{}}\PY{n}{Fore}\PY{o}{.}\PY{n}{LIGHTRED\PYZus{}EX}\PY{l+s+si}{\PYZcb{}}\PY{l+s+si}{\PYZob{}}\PY{n+nb}{round}\PY{p}{(}\PY{n}{y\PYZus{}0}\PY{p}{,}\PY{+w}{ }\PY{l+m+mi}{2}\PY{p}{)}\PY{l+s+si}{\PYZcb{}}\PY{l+s+si}{\PYZob{}}\PY{n}{Style}\PY{o}{.}\PY{n}{RESET\PYZus{}ALL}\PY{l+s+si}{\PYZcb{}}\PY{l+s+s2}{\PYZdq{}}\PY{p}{,} \PY{n}{end}\PY{o}{=}\PY{l+s+s2}{\PYZdq{}}\PY{l+s+s2}{: }\PY{l+s+s2}{\PYZdq{}}\PY{p}{)}
    \PY{n}{approx}\PY{p}{,} \PY{n}{\PYZus{}}\PY{p}{,} \PY{n}{num\PYZus{}iterations} \PY{o}{=} \PY{n}{newtons}\PY{o}{.}\PY{n}{newtons\PYZus{}method}\PY{p}{(}
        \PY{n}{f}\PY{p}{,} \PY{n}{f\PYZus{}prime}\PY{p}{,} \PY{n}{y\PYZus{}0}\PY{p}{,} \PY{n}{eps}\PY{p}{,} \PY{l+m+mi}{9999999}\PY{p}{,} \PY{n}{generate\PYZus{}table}\PY{o}{=}\PY{k+kc}{False}\PY{p}{,} \PY{n}{log}\PY{o}{=}\PY{k+kc}{False}\PY{p}{)}
    
    \PY{n}{ans} \PY{o}{=} \PY{l+s+sa}{f}\PY{l+s+s2}{\PYZdq{}}\PY{l+s+s2}{ }\PY{l+s+si}{\PYZob{}}\PY{n}{Fore}\PY{o}{.}\PY{n}{LIGHTGREEN\PYZus{}EX}\PY{l+s+si}{\PYZcb{}}\PY{l+s+si}{\PYZob{}}\PY{n}{approx}\PY{l+s+si}{:}\PY{l+s+s2}{.10f}\PY{l+s+si}{\PYZcb{}}\PY{l+s+si}{\PYZob{}}\PY{n}{Style}\PY{o}{.}\PY{n}{RESET\PYZus{}ALL}\PY{l+s+si}{\PYZcb{}}\PY{l+s+s2}{ in }\PY{l+s+si}{\PYZob{}}\PY{n}{Fore}\PY{o}{.}\PY{n}{LIGHTMAGENTA\PYZus{}EX}\PY{l+s+si}{\PYZcb{}}\PY{l+s+si}{\PYZob{}}\PY{n}{num\PYZus{}iterations}\PY{l+s+si}{\PYZcb{}}\PY{l+s+s2}{ iterations}\PY{l+s+si}{\PYZob{}}\PY{n}{Style}\PY{o}{.}\PY{n}{RESET\PYZus{}ALL}\PY{l+s+si}{\PYZcb{}}\PY{l+s+s2}{.}\PY{l+s+s2}{\PYZdq{}}

    \PY{n+nb}{print}\PY{p}{(}\PY{n}{ans}\PY{p}{)}
\end{Verbatim}
\end{tcolorbox}

    \begin{Verbatim}[commandchars=\\\{\}]
\textcolor{ansi-cyan}{y\_0 = }\textcolor{ansi-red-intense}{0.05}:  \textcolor{ansi-green-intense}{0.0370037459} in \textcolor{ansi-magenta-intense}{7 iterations}.
\textcolor{ansi-cyan}{y\_0 = }\textcolor{ansi-red-intense}{0.06}:  \textcolor{ansi-green-intense}{0.0370037448} in \textcolor{ansi-magenta-intense}{8 iterations}.
\textcolor{ansi-cyan}{y\_0 = }\textcolor{ansi-red-intense}{0.07}:  \textcolor{ansi-green-intense}{0.0370037447} in \textcolor{ansi-magenta-intense}{8 iterations}.
\textcolor{ansi-cyan}{y\_0 = }\textcolor{ansi-red-intense}{0.08}:  \textcolor{ansi-green-intense}{0.0370037447} in \textcolor{ansi-magenta-intense}{8 iterations}.
\textcolor{ansi-cyan}{y\_0 = }\textcolor{ansi-red-intense}{0.09}:  \textcolor{ansi-green-intense}{0.0370037448} in \textcolor{ansi-magenta-intense}{8 iterations}.
\textcolor{ansi-cyan}{y\_0 = }\textcolor{ansi-red-intense}{0.1}:  \textcolor{ansi-green-intense}{0.0370037452} in \textcolor{ansi-magenta-intense}{8 iterations}.
\textcolor{ansi-cyan}{y\_0 = }\textcolor{ansi-red-intense}{0.11}:  \textcolor{ansi-green-intense}{0.0370037459} in \textcolor{ansi-magenta-intense}{8 iterations}.
\textcolor{ansi-cyan}{y\_0 = }\textcolor{ansi-red-intense}{0.12}:  \textcolor{ansi-green-intense}{0.0370037448} in \textcolor{ansi-magenta-intense}{9 iterations}.
\textcolor{ansi-cyan}{y\_0 = }\textcolor{ansi-red-intense}{0.13}:  \textcolor{ansi-green-intense}{0.0370037447} in \textcolor{ansi-magenta-intense}{9 iterations}.
\textcolor{ansi-cyan}{y\_0 = }\textcolor{ansi-red-intense}{0.14}:  \textcolor{ansi-green-intense}{0.0370037447} in \textcolor{ansi-magenta-intense}{9 iterations}.
\textcolor{ansi-cyan}{y\_0 = }\textcolor{ansi-red-intense}{0.15}:  \textcolor{ansi-green-intense}{0.0370037440} in \textcolor{ansi-magenta-intense}{6 iterations}.
\textcolor{ansi-cyan}{y\_0 = }\textcolor{ansi-red-intense}{0.16}:  \textcolor{ansi-green-intense}{0.0370037457} in \textcolor{ansi-magenta-intense}{9 iterations}.
\textcolor{ansi-cyan}{y\_0 = }\textcolor{ansi-red-intense}{0.17}:  \textcolor{ansi-green-intense}{0.0370037447} in \textcolor{ansi-magenta-intense}{10 iterations}.
\textcolor{ansi-cyan}{y\_0 = }\textcolor{ansi-red-intense}{0.18}:  \textcolor{ansi-green-intense}{0.0370037447} in \textcolor{ansi-magenta-intense}{10 iterations}.
\textcolor{ansi-cyan}{y\_0 = }\textcolor{ansi-red-intense}{0.19}:  \textcolor{ansi-green-intense}{0.0370037451} in \textcolor{ansi-magenta-intense}{9 iterations}.
\textcolor{ansi-cyan}{y\_0 = }\textcolor{ansi-red-intense}{0.2}:  \textcolor{ansi-green-intense}{0.0370037448} in \textcolor{ansi-magenta-intense}{11 iterations}.
\textcolor{ansi-cyan}{y\_0 = }\textcolor{ansi-red-intense}{0.21}:  \textcolor{ansi-green-intense}{0.0370037447} in \textcolor{ansi-magenta-intense}{11 iterations}.
\textcolor{ansi-cyan}{y\_0 = }\textcolor{ansi-red-intense}{0.22}:  \textcolor{ansi-green-intense}{0.0370037453} in \textcolor{ansi-magenta-intense}{11 iterations}.
\textcolor{ansi-cyan}{y\_0 = }\textcolor{ansi-red-intense}{0.23}:  \textcolor{ansi-green-intense}{0.0370037447} in \textcolor{ansi-magenta-intense}{12 iterations}.
\textcolor{ansi-cyan}{y\_0 = }\textcolor{ansi-red-intense}{0.24}:  \textcolor{ansi-green-intense}{0.0370037458} in \textcolor{ansi-magenta-intense}{12 iterations}.
\textcolor{ansi-cyan}{y\_0 = }\textcolor{ansi-red-intense}{0.25}:  \textcolor{ansi-green-intense}{0.0370037444} in \textcolor{ansi-magenta-intense}{12 iterations}.
    \end{Verbatim}

    As seen above, as the initial estimate, \(y_0\) is increased beyond
\(0.05\) (up to \(0.25\)), more iterations of Newton's method are
required in order to achieve the same level of accuracy. In other words,
for the same iteration number, a solution starting with a higher \(y_0\)
has a lower accuracy.

The difference in the time taken to solve is negligable here, however,
so is not reported.

    \subsection{Question 4}\label{question-4}

In the Constant Growth DDM model, the present value of the share is

\[PV = \sum_{t=1}^{\infty} \frac{D_t}{(1 + k)^t}\]

where \(D_1, D_2, ...\) are (non-random) dividends and \(k > 0\) is the
required rate of return

Suppose \(D_0 > 0\), \(k > 0\) and \(g > 0\)

Derive the formula for the present value (2) when

\[D_t = D_0 (1 + g)^{\lceil \frac{t}{2} \rceil}, \hspace{0.5cm} t = 1, 2, ...\]

where \(\lceil x \rceil\) is the smallest integer greater than or equal
to \(x\). What is the condition of \(g\) so that \(PV\) is finite? To
get full marks, you will need to write an explicit expression (without
summation).

    First substitute for \(D_t\) as defined in the question and expand to
see the pattern

\begin{align}
    PV &= \sum_{t=1}^{\infty} \frac{D_t}{(1 + k)^t} \\
    &= \sum_{t=1}^{\infty} \frac{D_0 (1 + g)^{\lceil \frac{t}{2} \rceil}}{(1 + k)^t} \\ 
    &= \left[ D_0 \cdot \frac{(1 + g)^{\lceil \frac{1}{2} \rceil}}{(1 + k)^1} \right] + \left[ D_0 \cdot \frac{(1 + g)^{\lceil \frac{2}{2} \rceil}}{(1 + k)^2} \right] + \left[ D_0 \cdot \frac{(1 + g)^{\lceil \frac{3}{2} \rceil}}{(1 + k)^3} \right] + \left[ D_0 \cdot \frac{(1 + g)^{\lceil \frac{4}{2} \rceil}}{(1 + k)^4} \right] + \cdots \\
    &= \left[ D_0 \cdot \frac{(1 + g)^{1}}{(1 + k)^1} \right] + \left[ D_0 \cdot \frac{(1 + g)^{1}}{(1 + k)^2} \right] + \left[ D_0 \cdot \frac{(1 + g)^{2}}{(1 + k)^3} \right] + \left[ D_0 \cdot \frac{(1 + g)^{2}}{(1 + k)^4} \right] + \cdots
\end{align}

Now consider, splitting up the geometric series into sub series where

\begin{itemize}
\tightlist
\item
  the exponent on the numerator is half the exponent on the denominator;
  and
\item
  the above is not the case
\end{itemize}

We then have

\begin{align}
    PV &= \left\{ \left[ D_0 \cdot \frac{(1 + g)^{1}}{(1 + k)^2} \right] + \left[ D_0 \cdot \frac{(1 + g)^{2}}{(1 + k)^4} \right] + \cdots \right\} + \left\{ \left[ D_0 \cdot \frac{(1 + g)^{1}}{(1 + k)^1} \right] + \left[ D_0 \cdot \frac{(1 + g)^{2}}{(1 + k)^3} \right] + \cdots \right\} \\
    &=  D_0 \left\{ \left[ \frac{(1 + g)^{1}}{(1 + k)^2} \right] + \left[ \frac{(1 + g)^{2}}{(1 + k)^4} \right] + \cdots \right\} + D_0 \left\{ \left[ \frac{(1 + g)^{1}}{(1 + k)^1} \right] + \left[ \frac{(1 + g)^{2}}{(1 + k)^3} \right] + \cdots \right\} \\
    &=  D_0 \left\{ \left[ \frac{(1 + g)^{1}}{(1 + k)^2} \right] + \left[ \frac{(1 + g)^{2}}{(1 + k)^4} \right] + \cdots \right\} + D_0 (1 + k) \left\{ \left[ \frac{(1 + g)^{1}}{(1 + k)^2} \right] + \left[ \frac{(1 + g)^{2}}{(1 + k)^4} \right] + \cdots \right\} \\
    &= \sum_{t=1}^{\infty} D_0 \left[ \frac{1 + g}{(1 + k)^2} \right]^{t} + \sum_{t=1}^{\infty} D_0 (1 + k) \left[ \frac{1 + g}{(1 + k)^2} \right]^{t} \\
    &= \sum_{t=1}^{\infty} D_0 \frac{1 + g}{(1 + k)^2} \left[ \frac{1 + g}{(1 + k)^2} \right]^{t-1} + \sum_{t=1}^{\infty} D_0 (1 + k) \frac{1 + g}{(1 + k)^2} \left[ \frac{1 + g}{(1 + k)^2} \right]^{t-1} \\
    &= \sum_{t=1}^{\infty} D_0 \frac{1 + g}{(1 + k)^2} \left[ \frac{1 + g}{(1 + k)^2} \right]^{t-1} + \sum_{t=1}^{\infty} D_0 \frac{1 + g}{1 + k} \left[ \frac{1 + g}{(1 + k)^2} \right]^{t-1} \\
\end{align}

These are both valid geometric series. Now apply the infinite geometric
series formula

\[\sum_{n=1}^{\infty} a r^{n - 1} = S_n =\frac{a}{1-r}\]

\begin{align}
    PV &= \frac{D_0 \frac{1 + g}{(1 + k)^2}}{1 - \frac{1 + g}{(1 + k)^2}} + \frac{D_0 \frac{1 + g}{1 + k}}{1 - \frac{1 + g}{(1 + k)^2}} \\
    &= \frac{D_0 \frac{1 + g}{(1 + k)^2} + D_0 \frac{1 + g}{1 + k}}{1 - \frac{1 + g}{(1 + k)^2}} \\
    &= \frac{D_0 \frac{1 + g}{1 + k} \left[\frac{1}{1 + k} + 1\right]}{1 - \frac{1 + g}{(1 + k)^2}} \\
\end{align}

    Condition of \(g\) for when \(PV\) is finite.

From the geometric series, \(PV\) will be finite when \(|r| < 1\), where
\(r = \frac{1 + g}{(1 + k)^2}\). So solve

\begin{align}
    |r| &< 1 \\
    \left|\frac{1 + g}{(1 + k)^2} \right| &< 1 \\
    -(1 + k)^2 < 1 + g &< (1 + k)^2 \\
    -(1 + 2k + k^2) - 1 < g &< (1 + 2k + k^2) - 1 \\
    -(2 + 2k + k^2) < g &< k^2 + 2k \\
    -2 - 2k - k^2 < g &< k(k + 2) \\
\end{align}

Since \(g > 0\) and \(k > 0\), we do not need to consider the lower
bound. Thus for \(PV\) to be finite (\(PV < \infty\)), the following
condition must hold

\[g < k(k + 2)\]


    % Add a bibliography block to the postdoc
    
    
    
\end{document}
